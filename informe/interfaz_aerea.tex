\documentclass[conference,a4paper,10pt,oneside,final]{tfmpd}

\usepackage[spanish]{babel}
\usepackage[utf8]{inputenc}
\usepackage{graphicx}           % inserción de graficos

\begin{document}

	\title{Interfaz aérea}
	\author{Bedrij Walter, Benitez Federico y Benitez Fernando \\
	\textit{Trabajo práctico final de ``Procesamiento digital de imágenes'', II-FICH-UNL.}}
	\markboth{PROCESAMIENTO DIGITAL DE IMÁGENES: TRABAJO FINAL}{}
	\date{}

	\maketitle

    \begin{abstract}
   	 En este trabajo se desarrolla un algoritmo para la localización
   	 del dedo índice en tiempo real,
   	 para así usarse como interfaz aérea sin marcadores.
   	 Se hicieron pruebas en ambientes con distinta iluminación
   	 para hallar los parámetros más convenientes para la segmentación.
   	 Se detalla tambien las virtudes y deficiencias de distintas variables analizadas.

    \end{abstract}
	\begin{keywords}
		 Interfaz aérea, segmentación, marcadores, object-tracking
	\end{keywords}

    \section{Introducción}
   	 Según la literatura a estos tipos de problemas se los aborda
   	 con la utilización de un patrón que sea facilmente separable del resto.
   	 Sin embargo,aca se descarta esta aproximación tan molesta para el usuario,
   	 permitiéndole utilizar sus manos desnudas.

   	 El procesamiento es como sigue:
   	 \begin{enumerate}
   		 \item Obtención de una imagen de referencia para el fondo
   		 \item Captura de los fotogramas del vídeo
   		 \item Enmascaramiento de la piel
   		 \item Eliminación de la cara
   		 \item Ubicación del puntero
   	 \end{enumerate}

    \section{Materiales y métodos}
   	 \subsection{Obtención de una imagen de referencia}
   		 \emph{Motivación} eliminar objetos estáticos del análisis.
   		 Permitir que la mano se sitúe en zonas de colorización parecida
   		 sin mayores problemas.

   		 Esto consistió en primero esperar la finalización del proceso automático
   		 de calibración de la cámara. Luego se procedió a tomar la instantánea,
   		 tomando precaución de no alterar mediante sombras

   		 Con respecto a la iluminación, esta se considera constante e incidente,
   		 evitando también el lavado de los colores.

   	 \subsection{Captura de los fotogramas del vídeo}
   		 Cada fotograma es tratado de manera independiente.
   		 Se adquiere una imagen a color,

   	 \subsection{Enmascaramiento de la piel}
   		 En esta etapa se analizaron parámetros que pudieran hacer
   		 más factible la separación del objeto de interés.
   		 Como ser el hue, la saturación, la intensidad, y combinaciones entre las mismas
   		 
   		 Se observó que la saturación era muy dependiente de la iluminación, por lo que fue descartada.
   		 En el caso de la intensidad,el objeto de interes se perdia con el fondo ante cambios de posicion,por ende no se avanzo en esta linea
   		 Sin embargo, el hue era un parametro robusto ante distintos grados de iluminacion,
   		 es decir, presentaba poco ruido de sal y huecos internos en el objeto, como asi tambien objetos que tenian hue muy proximos a la mano,
   		 Afortunadamente, por ser inmoviles,podian ser removibles con la primera etapa de procesamiento.
   		 
   		 Tambien se analizo con combinaciones logicas entre los canales, pero debido a las deventajas ya nombradas
   		 no se tuvo mayor exito que hue solo.

   		 Se aplica un filtro de mediana para eliminar el ruido impulsivo en el hue.
   		 Se realiza un umbralizado de intervalo en el rango del hue de la mano $H=7$ con un tamaño de $\alpha \pm 5$

   		 Para eliminar los huecos internos en el hue, se procedio aplicar una dilatacion con una mascara de 5x5,esto provoco un grado de deformacion
   		 en los objetos, luego para recuperar medianamente la forma original, se procedio a una erosion con mascara de 5x5

   		 Se combina con la máscara de diferencia, obteniéndose entonces
   		 objetos no pertenecientes al fondo con un tono parecido al de la piel.

   		 \subsubsection{Crecimiento inverso de regiones}
   			 Con el objeto de obtener regiones simples se realiza el crecimiento inverso.
   			 A partir de un punto que se conoce fuera de cualquier zona de interés,
   			 se cosidera a todo la región perteneciente a ese punto como fondo.


    \subsection{Eliminación de la cara}
   		 Luego del proceso anterior, se tiene una imagen compuesta por objetos
   	 tipo piel. Estas pueden ser la cara, las manos y algún ruido que haya pasado los filtros.

   		 Por eso se etiquetan las regiones, quedándose con las $3$ de mayor área.
   	 Que se supone pertenecientes a ambas manos y la cara.
   	 Por observación debieran de tener un área similar, si resulta que alguna
   		 es excesivamente pequeña con respecto a las otras ($25\%$) es considerada
   	 como ruido y eliminada.

   	 
   		 Queda discriminar las correspondientes a las manos.
   	 Es decir, eliminar la región que identifica a la cara.

   	 Para facilitar el proceso, la mano debe tener un dedo extendido.
   	 Suponiendo a la cara como redonda, es de esperar que al hacer
   	 el coeficiente
   		 \[C = \frac{\mbox{Área}}{r^2}\]
   	 este sea cercano a $1$. Donde $r$ es el radio del círculo cuyo centro
   	 se corresponde al centroide de la región y pasa por su punto más alejado.

    \subsection{Ubicación del puntero}
   	 Ya se tienen las áreas interesantes.
   		 Se ubica al puntero como al punto más lejano del centroide de la región.
	 Para suavizar la trayectoria del puntero, se le aplica un filtro de Kalman.
	Se pueden observa que los punteros son independientes, es decir el usuario 
	sacar de escena  una de las 2 manos.

\subsection{Diseño de interfaz}
  Para implementaciónon de la interfaz de la interfaz se diseño una paleta de colores que consiste en 5 colores : rojo, verde, azul, blanco y negro. También se diseñaron dos objetos, el borrador y el visor de color actual.
La manipulación consiste en tener el dedo indice bien extendido, el dedo indice izquierdo se utiliza para la seleccion del color y del borrador, el dedo indice drecho es el puntero para dibujar.

	
\section{Resultados}
		El uso de una imagen de referencia para el fondo permitió que la mano se ubicara sobre zonas de tonalidad parecida.
		Sin embargo, la cámara utilizada tenía función de autofoco, por lo que se creaban falsas diferiencias. 
Una de las aproximaciones para tratar este problema fue la de comparar contra la imagen anterior, pero se desechó rápidamente al notar que podrían surgir nuevas regiones al ocultar y descubrir el fondo.
Se optó finalmente por incrementar el umbral de lo que se consideraba como cambio, dejándolo en $30/255$

		El hue mostró ser bastante robusto a los cambios de intensidad de iluminación, por lo que pudo fijarse un umbral en forma experimental y no se necesitaba de una posterior calibración. 
Éste fue probado con dos tipos de piel, usando iluminación por lámpara incandescente y fluorescente (compacta).
		%Sin embargo, se tuvo problemas con el ajuste de blancos de la cámara,
		%tomando la piel clara una tonalidad azulada en vez de rojiza al usar lámparas incandescentes. (discutir)

	La lámpara incandescente producía artefactos en el hue, detectándose como válida la zona de mayor intensidad. 
	
	En la detección de la mano, el mayor problema lo presentaba un enmascaramiento defectuoso. En caso de que el dedo se considerara como una región separada seria eliminado por su área insignificante, quedando solo la palma
de forma aproximadamente circular, por lo que la heurística fallaba.
	% que fede tire algo después por acá
	La heurística permite independencia de la escala, pudiendo entonces usar la interfaz tanto de forma cercana (centado frente a la pc) como a distancia (2m) %probar

		Con respecto a la interfaz observamos unas oscilaciones instantaneas producto de detecciones de puntos lejanos no deseados con respecto al centroide de las manos como en el caso cuando se usa la interfaz con una remera corta, como este método se basa en diferencias de imagenes y hue, el brazo sobrevive a los procesamientos  de eliminacion, y como es aproximadamente del mismo color de la mano, es segmentado obteniendo asi puntos lejanos no deseados, por ello una de las restricciones es usar  atuendo que oculte el brazo y descubra las manos.


\end{document}

